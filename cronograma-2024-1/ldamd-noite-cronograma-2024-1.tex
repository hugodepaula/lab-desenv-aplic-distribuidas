%%%%%%%%%%%%%%%%%%%%%%%%%%%%%%%%%%%%%%%
% Cronograma de aula
%%%%%%%%%%%%%%%%%%%%%%%%%%%%%%%%%%%%%%%

\documentclass[11pt,brazil, a4paper, fullpage]{article}


\newcommand\UNIVERSIDADE{PONTIFÍCIA  UNIVERSIDADE  CATÓLICA  DE  MINAS  GERAIS}
\newcommand\UNIDADE{Praça da Liberdade}
\newcommand\INSTITUTO{Inst. de Ciências Exatas e Informática}
\newcommand\CURSO{Eng. de Software}
\newcommand\PROFESSOR{Hugo de Paula}
\newcommand\EMAILPROFESSOR{hugo@pucminas.br}
\newcommand\DEPARTAMENTO{Departamento de Ciência da Computação}

\newcommand\DISCIPLINA{Lab. de Desenv. de Aplicações Móveis e Distribuídas}
\newcommand\ANO{2024}
\newcommand\SEMESTRE{1}
\newcommand\TURNO{noite}
\newcommand\PERIODO{5}

\def\year{\ANO}

\input{infos/some-definitions}
\input{infos/calpreambulo}

\newcommand{\DIAUM}{\Quarta}
\newcommand{\DIADOIS}{\Quarta}


\begin{document}

	\selectlanguage{brazil}

	\logoPUC{infos/puclogo_small_bw}
	\dadosDisciplina{\DISCIPLINA}{\CURSO}{\TURNO}{\ANO}{\SEMESTRE}{\PROFESSOR}{\EMAILPROFESSOR}

	%\dadosAvancadosDisciplina{04}{00}{04}{80}{04}{infos/objetivos.tex}{infos/ementa.tex}{3 $\times$ 25}{25}{infos/avaliacao.tex}{infos/biblio.tex}

	\calVisualColorido{1}{\DIAUM}{\DIADOIS}{\ANO}{07-07} % semestre, dias de aula (DOIS vezes por semana)
	%\calVisualColorido{2}{\DIAUM}{\DIADOIS}{\ANO}{12-21} % semestre, dias de aula (DOIS vezes por semana)


\begin{center}

\small
%\footnotesize
\begin{calendar}{2/5/\ANO}{22} % Semestre comeca no dia 1 de agosto, e dura por 21 semanas.
\setlength{\calboxdepth}{.3in}
\setlength{\calwidth}{0.95\textwidth}

%
%% configuracao da semana
\semanaQua
%
%% lista de Aulas
\caltexton{1}{Apresentação da disciplina e Nivelamento de Redes}
\caltextnext{Programação Socket e Multicast em Java}
\caltextnext{Programação Java RMI}
\caltextnext{Programação Java RMI}
\caltextnext{Comunicação indireta}
\caltextnext{Comunicação indireta}
\caltextnext{Comunicação indireta}
\caltextnext{Desenvolvimento web MVC}
\caltextnext{Desenvolvimento web MVC}
\caltextnext{Desenvolvimento web MVC}
\caltextnext{Desenvolvimento móvel com Flutter}
\caltextnext{Desenvolvimento móvel com Flutter}
\caltextnext{Desenvolvimento móvel com Flutter}
\caltextnext{Desenvolvimento móvel com Flutter}
\caltextnext{Desenvolvimento móvel com Flutter}
\caltextnext{Serverless computing}
\caltextnext{Serverless computing}
\caltextnext{Serverless computing}
%\caltextnext{Acompanhamento}
\caltextnext{Apresentação}
%\caltextnext{Reavaliação}
%
% feriados e avisos
% Se n?o quiser que os avisos sejam colocados no cronograma, basta comentar o comando input.
\input{infos/feriados}
%%\input{infos/avisos}
%

%\aviso{9/19/2022}{AULA ADICIONAL SEGUNDA-FEIRA, DIA 19/09, 10h40}
%\aviso{10/12/2022}{AULA ADICIONAL QUINTA-FEIRA, DIA 13/10, 8h50}
%\aviso{10/31/2022}{AULA ADICIONAL SEGUNDA-FEIRA, DIA 19/09, 10h40}


\end{calendar}
\end{center}

\end{document}
